\chapter{Conclusion and Future Work}

\section{Conclusion}
The rapid growth in the number of cell phones and other wireless devices has 
brought about a scarcity of spectral resources. This is further worsened by the
inefficient usage of the spectrum. Cognitive Radio has a lot of potential to solve this problem
by finding the spectrum holes and enabling secondary users to utilize them. CR 
thereby increases the number of mobile device users possible by putting 
underutilized frequency bands into use. We have 
demonstrated the capabilities of CR by developing a 2-frequency CR Test-Bed and
a 4-frequency CR Test-Bed and successfully testing them out.  We have shown
that the secondary users yield the frequency channel they have been utilizing
for communication as soon as the presence of primary users is detected thus
giving a higher priority to the primary users.


\section{Future work}
In the test-beds that we developed, we have used energy detection based spectrum sensing
technique. We are measuring the energy at the uplink frequency of the primary BTS 
to detect the presence of active primary users. But this sensed energy level is dependent
on the distance of the primary users from the sensing unit. So, this makes our 
decision-making criteria for the presence of primary users on a particular 
frequency band makes distance dependent. 
In future, this dependency on
distance of primary users from the sensor can be removed by resorting to better 
spectrum sensing techniques.

Also, as soon as the primary users emerge on the scene, the secondary calls 
are dropped and has to be restarted after switching to a new frequency. Better
algorithms can be developed to avoid this call drop.