\chapter{Conclusion and Future Work}

\section{Conclusion}
The rapid growth in the number of cell phones and other wireless devices has 
brought about a scarcity of spectral resources. This is further worsened by the
inefficient usage of the spectrum. \acrlong{cr} has a lot of potential to solve this problem
by finding the spectrum holes and enabling \glspl{su} to utilize them. \gls{cr} 
thereby increases the number of mobile device users possible by putting 
underutilized frequency bands into use. We have 
demonstrated the capabilities of \gls{cr} by developing a 2-frequency \gls{cr} Test-Bed and
a 4-frequency \gls{cr} Test-Bed and successfully testing them out.  We have shown
that the \glspl{su} yield the frequency channel they have been utilizing
for communication as soon as the presence of \glspl{pu} is detected thus
giving a higher priority to the \glspl{pu}.


\section{Future work}
To detect the presence of active \glspl{pu}, we have used energy detection 
based spectrum sensing method to measure the energy level at the uplink 
frequency of the primary \gls{bts}. This energy level is dependent
on the distance of the \glspl{pu} from the sensing unit. Thus, our criteria
for determining the presence of \glspl{pu} is distance dependent. 
In future, this dependency could be removed by resorting to better 
spectrum sensing techniques.

When our secondary system switches to a new underutilized frequency band, the
secondary calls get dropped and have to be restarted. Better
algorithms could be designed to avoid this call drop from happening.