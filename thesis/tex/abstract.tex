\chapter*{\centering{Abstract}}
The explosive proliferation of mobile phones and other wireless communication
devices has rendered the spectrum an insufficient resource. To make things
worse, there exists a widespread underutilization of the spectrum. Cognitive
Radio (CR) offers to tackle this problem by finding the unused frequency bands
in the spectrum (also known as spectrum holes) and allocating them to the
secondary users (SUs) for use. Hence, CR makes the usage of spectrum more
efficient.

We demonstrate the capabilities of CR by developing a Two-frequency CR Test-bed
and a Four-frequency CR Test-bed. Primary Users (PUs) and SUs are made to
coexist in the same frequency band. Energy detection method is used for the
spectrum sensing. SUs switch from their frequency band of operation to an
unused frequency band as soon as the activity of PUs is detected.