\chapter{Codes}

\section{Code for the 2-frequency system}


\subsection{freq2secondaryBTS.py}

This code was written to demonstrate the 2-frequency system.
This code was written by modifying the already available program named 
\textsf{usrp\_spectrum\_sense.py'} that comes together with the GNURadio software
package. We set the 
default UHD device address to \textsf{192.168.20.2} because that is the IP address
of the USRP device we are using as a spectrum sensor. The default samping rate
was set to 1e6 i.e. 1MHz. The default FFT size is given as
\textsf{sampling rate /channel bandwidth}. We wanted an FFT size of 1024 so we
set the bandwidth to
976.56 Hz since 1 MHz / 976.56 Hz $\approx$ 1024.

The class 'my\_top\_block' was modified by replacing the lines:
\begin{lstlisting}[language=Python]
        self.channel_bandwidth = options.channel_bandwidth
        self.min_freq = eng_notation.str_to_num(args[0])
        self.max_freq = eng_notation.str_to_num(args[1])
        if self.min_freq > self.max_freq:
            self.min_freq, self.max_freq = self.max_freq, self.min_freq    
\end{lstlisting}
with the lines
\begin{lstlisting}[language=Python]
        self.channel_bandwidth = options.channel_bandwidth
        self.down_freq = eng_notation.str_to_num(args[0])
        self.up_freq = (self.down_freq) - 45e6    
\end{lstlisting}

The method \textsf{set\_next\_freq} of the class \textsf{my\_top\_block} was modified by
replacing
\begin{lstlisting}[language=Python]
        target_freq = self.next_freq
        self.next_freq = self.next_freq + self.freq_step
        if self.next_freq >= self.max_center_freq:
            self.next_freq = self.min_center_freq
\end{lstlisting}
with
\begin{lstlisting}[language=Python]
        target_freq = self.up_freq
\end{lstlisting}




In the code listing that follows we have listed only the functions that we
customized and the ones that we added.

\begin{lstlisting}[language=Python]
def main_loop(tb):
    startOpenBTS(tb.down_freq,tb)

def sub_loop(tb):
    print 'fft size', tb.fft_size
    N = tb.fft_size
    mid = N // 2
    cusum = 0
    counter = 0

    while 1:
        m = parse_msg(tb.msgq.delete_head())
        center_freq = m.center_freq
        bins = 102
        power_data = 0
        noise_floor_db = 0 
        for i in range(1, bins+1):
            power_data += m.data[mid-i] + m.data[mid+i]
        power_data += m.data[mid]
        power_data /= ((2*bins) + 1)
        power_db = 10*math.log10(power_data/tb.usrp_rate) - noise_floor_db
        power_threshold = -59.0
        print datetime.now(), "center_freq", center_freq, "power_db", power_db
        cusum = max(0, cusum + power_db - power_threshold)
        if (cusum > 0):
            counter += 1
            if (counter > 2):
                print "CUSUM is now positive!!!"
                down_freq = center_freq + 45e6
                quitOpenBTS(down_freq, tb)
                break

def startOpenBTS(downFrequency,tb):
    arfcn=int((downFrequency-935e6)/2e5)
    if (arfcn < 0):
        print "ARFCN must be > 0 !!!"
        sys.exit(1)
    print 'ARFCN=', arfcn
    t=(arfcn,)
    conn=sqlite3.connect("/etc/OpenBTS/OpenBTS.db")
    cursor=conn.cursor()
    cursor.execute("update config set valuestring=? where keystring='GSM.Radio.C0'",t)
    conn.commit()
    f=subprocess.Popen(os.path.expanduser('~/ddpOpenBTS/runOpenBTS.sh'))
    f.wait()
    tb.msgq.delete_head()
    time.sleep(0.25)
    sub_loop(tb)


def quitOpenBTS(downFreq, tb):
    f=subprocess.Popen(os.path.expanduser('~/ddpOpenBTS/quitOpenBTS.sh'))
    f.wait()
    if downFreq <= 945e6:
        newDownFreq = downFreq + 10e6
    else:
        newDownFreq = downFreq - 10e6
    tb.up_freq = newDownFreq - 45e6
    print "new tb.up_freq: ", tb.up_freq
    startOpenBTS(newDownFreq, tb)
\end{lstlisting}


\section{Code for the 4-frequency system}
\subsection{secondaryBTS.py}

Most of the code is similar to \textsf{freq2secondaryBTS.py}. The only modified 
functions are listed below:

\begin{lstlisting}[language=Python]
def main_loop(tb):
    startOpenBTS(tb.down_freq,tb)

def sub_loop(tb):
    print 'fft size', tb.fft_size
    N = tb.fft_size
    mid = N // 2
    cusum = 0
    counter = 0
    
    while 1:
        m = parse_msg(tb.msgq.delete_head())
        center_freq = m.center_freq
        bins = 102
        power_data = 0
        noise_floor_db = 0
        for i in range(1, bins+1):
            power_data += m.data[mid-i] + m.data[mid+i]
        power_data += m.data[mid]
        power_data /= ((2*bins) + 1)
        power_db = 10*math.log10(power_data/tb.usrp_rate) - noise_floor_db
        power_threshold = -70.0
        print datetime.now(), "center_freq", center_freq, "power_db", power_db
        cusum = max(0, cusum + power_db - power_threshold)
        if (cusum > 0):
            counter += 1
            if (counter > 2):
                print "CUSUM is now positive!!!"
                down_freq = center_freq + 45e6
                quitOpenBTS(down_freq, tb)
                break
        else:
            counter = 0

def startOpenBTS(downFrequency,tb):
    arfcn=int((downFrequency-935e6)/2e5)
    if (arfcn < 0):
        print "ARFCN must be > 0 !!!"
        sys.exit(1)
    print 'ARFCN=', arfcn
    t=(arfcn,)
    conn=sqlite3.connect("/etc/OpenBTS/OpenBTS.db")
    cursor=conn.cursor()
    cursor.execute("update config set valuestring=? where keystring='GSM.Radio.C0'",t)
    conn.commit()
    f=subprocess.Popen(os.path.expanduser('~/ddpOpenBTS/runOpenBTS.sh'))
    f.wait()
    tb.msgq.delete_head()
    time.sleep(0.25)
    sub_loop(tb)

def quitOpenBTS(downFreq, tb):
    f=subprocess.Popen(os.path.expanduser('~/ddpOpenBTS/quitOpenBTS.sh'))
    f.wait()    
    newDownFreq = getNewChannel(downFreq, tb)
    startOpenBTS(newDownFreq, tb)

def getNewChannel(downFreq, tb):
    newDownFreq = downFreq + 7e6
    if newDownFreq > 960e6:
        newDownFreq = 936e6
    tb.up_freq = newDownFreq - 45e6
    print "new tb.up_freq: ", tb.up_freq
    tb.msgq.delete_head()
    time.sleep(0.25)
    print 'fft size', tb.fft_size
    N = tb.fft_size
    mid = N // 2
    cusum = 0
    counter = 0
    loopcounter = 0

    while loopcounter < 10:
        m = parse_msg(tb.msgq.delete_head())
        center_freq = m.center_freq
        bins = 102
        power_data = 0
        for i in range(1, bins+1):
            power_data += m.data[mid-i] + m.data[mid+i]
        power_data += m.data[mid]
        power_data /= ((2*bins) + 1)
        power_db = 10*math.log10(power_data/tb.usrp_rate)
        power_threshold = -70.0
        print datetime.now(), "center_freq", center_freq, "power_db", power_db
        print "precheck"
        cusum = max(0, cusum + power_db - power_threshold)
        loopcounter += 1
        if (cusum > 0):
            counter += 1
            if (counter > 2):
                print "CUSUM is now positive!!!"
                newDownFreq = getNewChannel(newDownFreq, tb)
                break
        else:
            counter = 0
    return newDownFreq
\end{lstlisting}




\section{primaryBTS.py}
\begin{lstlisting}[language=Python]
#!/usr/bin/env python
import sys
import sqlite3
import os
import re

def main_loop():
    usage = "usage: %prog channel_freq"
    if len(sys.argv) != 2:
        print 'usage:', sys.argv[0], 'channel_freq'
        sys.exit(1)
    center_freq = int(re.match(r'\d+', sys.argv[1]).group())*1e6
    startOpenBTS(center_freq)

def startOpenBTS(frequency):            
    arfcn=int((frequency-935e6)/2e5)
    print 'ARFCN=', arfcn
    t=(arfcn,)
    conn=sqlite3.connect("/etc/OpenBTS/OpenBTS.db")
    cursor=conn.cursor()
    cursor.execute("update config set valuestring=? where keystring='GSM.Radio.C0'",t)
    conn.commit()
    f=os.popen('~/ddpOpenBTS/runOpenBTS.sh')
    f.close()

if __name__ == '__main__':
    try:
        main_loop()
    except KeyboardInterrupt:
        pass
\end{lstlisting}


\section{runOpenBTS.sh}
\begin{lstlisting}[language=bash]
#!/bin/bash
sudo echo "Hi, this script starts OpenBTS in Ubuntu 12.04!"
sudo service asterisk restart
sudo gnome-terminal -x sh -c "sudo asterisk -r" &
cd ~/OpenBTS/openbts/trunk/apps/
sudo gnome-terminal --tab -e "sudo ../../../smqueue/trunk/smqueue/smqueue" \
    --tab -e "sudo ../../../subscriberRegistry/trunk/sipauthserve" &
sudo gnome-terminal --tab -e "sudo ./OpenBTS" \
    --tab -e "sudo ./OpenBTSCLI" &
cd ~
\end{lstlisting}


\section{quitOpenBTS.sh}
\begin{lstlisting}[language=bash]
#!/bin/bash
sudo echo "Hi, this script turns OpenBTS off in Ubuntu 12.04!"
sudo killall transceiver smqueue sipauthserve OpenBTSCLI asterisk
\end{lstlisting}

