\chapter{Spectrum sensing}

Spectrum sensing is the main task in the entire operation of cognitive
radio. Spectrum sensing is defined as the finding of spectrum holes in 
the local neighborhood of the \gls{cr} receiver. Spectrum holes are the
underutilized (in part or in full) subbands of spectrum at a particular time
in a specific location. Moreover for \gls{cr} to fulfil its potential
in solving the problem of spectrum underutilization, the spectrum sensing 
method used should be reliable and computationally feasible in real-time 
\cite{haykin09}.

There are many spectrum sensing techniques available. Three important ones of
them are as follows:
\begin{itemize}
    \item Energy detection
    \item Matched filter detection
    \item Cyclostationarity detection
\end{itemize}

\section{Energy detection}

Conventional energy detector is made up of a low pass filter, an A/D 
converter, a square law detector and an integrator (Figure 
\ref{energyDetection}a). This implementation is not flexible enough, 
especially in the case of narrowband signals and sinewaves. Also, this 
requires a pre-filter matched to the bandwidth $B$ of the signal to be scanned
\cite{cabric06}.

So, an alternative implementation is generally used where we find the squared 
magnitude of the \gls{fft}  using the Average Periodogram method (Figure 
\ref{energyDetection}b). In this architecture, we can alter the bandwidth of
frequencies scanned just by taking
the required number of \gls{fft}  bins. 

Spectrum sensing can be viewed as a binary hypothesis-testing problem 
\cite{zhang09}:
\begin{itemize}[noitemsep,topsep=0pt,parsep=0pt,partopsep=0pt]
    \item $H_0$: \acrlong{pu} is absent
    \item $H_1$: \acrlong{pu} is present
\end{itemize}
The detection is basically to decide between the following two hypotheses,
\begin{align}
    H_0 : \qquad x(t) &= n(t),  \nonumber \\
    H_1 : \qquad x(t) &= h(t)s(t) + n(t),  \nonumber
\end{align}
where $x(t)$ is the received signal, $s(t)$ is the \gls{pu} signal, $h(t)$
is the complex channel gain and $n(t)$ is the \gls{awgn} with zero mean 
and variance $\sigma_n^2$. Generally $h(t)$ is assumed
to be constant $h_0$ for the detection period. A statistics $Y$ is computed by
taking energy samples over a time $T$ in a bandwidth $B$ and compared with a 
predefined threshold $\gamma$ for making the decision.

Energy detection is one of the simplest methods of spectrum sensing. It is the 
optimal detection method for unknown signals. Moreover, it is widely used 
because its computationally less resource-intensive.

\begin{figure}
    \centering
    \includegraphics[width=0.7\textwidth]{../images/energyDetection}
    \caption[Energy Detection block diagram]{(a) Implementation using analog 
    filter and square law device \\
    (b) Implementation using periodogram. \\
    Source: {\cite{cabric06}}.}
    \label{energyDetection}
\end{figure}

But this method is not without problems. It is always to difficult to 
determine a threshold that will work for all situations. This method cannot
say whether an interfering signal is from a \gls{pu} or a \gls{su}.
Low \gls{snr} signals cannot be detected easily.

The frequency resolution can be improved by increasing N, the number of \gls{fft}  
points, but then this requires more samples and thereby takes more time.

\section{Matched filter detection}
A matched filter is a linear filter to maximize the output \gls{snr}  of a received 
signal. It is the optimum filter to detect signals that are known a priori 
\cite{wikiMF}.


In matched filtering, the received signal is first band pass filtered and then
convolved with the impulse response. The impulse response $h$ here is the 
reference signal itself \cite{bhatta11}. Matched filtering is so called 
because the impulse response is matched to the reference signal.
\begin{equation*}
    Y[n] = \sum_{-\infty}^{\infty} h[n-k]x[k]
\end{equation*}

Here, $x[k]$ is the received or unknown signal with additive noise.
The goal of matched filtering is to enhance the component of reference signal
in the received signal and to suppress the noise. This works best when the 
additive noise is completely orthogonal to the reference signal or when the
noise is completely Gaussian. In practice though, the noise doesn't turn out 
to be purely Gaussian.
 
Matched filtering requires only $O(1/SNR)$ samples to meet a given $P_d$, 
probability of detection requirement. Thus it requires less detection time.

But, matched filtering requires us to have a priori information about the 
received signal. This technique requires demodulating the received signal. For
demodulation, we require information like bandwidth, operating frequency, 
modulation type, pulse shaping, packet format, etc. Demodulating the received
signal correctly also requires timing synchronization, carrier 
synchronization, etc. It might still be possible to achive this because the
received data carry preambles, synchronization data, etc.

This method requires a specific type of receiver for every \gls{pu}.
Implementing this method on a receiver will increase the complexity and the 
power consumption greatly.

\begin{figure}
\centering
\includegraphics[width=0.8\textwidth]{../images/matchedFilter}
\caption[Matched Filter block diagram]{Block diagram of Matched Filter
implementation {\cite{kranthi13}}.}
\label{matchedFilter}
\end{figure}


\section{Comparison of sensing techniques}

\begin{figure}
\centering
\includegraphics[width=0.77\textwidth]{../images/compareSensing}
\caption[Comparison of sensing methods]{Comparison of some spectrum sensing 
methods {\cite{kranthi13}}}
\label{compareSensing}
\end{figure}

Energy detection technique is the simplest method of all but it is also the 
most error prone. On the other hand, matched filtering is the most complex but 
it is very accurate. Cyclostationarity detection is more complicated than 
energy detection but it is more accurate. There is no ideal detection method.
There are always compromises and tradeoffs to be made. Figure 
\ref{compareSensing} shows a comparison of some spectrum sensing methods.

There are other methods of spectrum sensing that are more involved. For
example multitaper spectral estimation, wavelet based detection, waveform
based detection, etc.


\section{Implementation of energy detection technique}

In this section, we give a brief mathematical overview of the Average 
Periodogram Analysis method, that we have used in this project to calculate 
the energy in various operating frequencies. We also dwell on the wideband 
spectrum analyzer which happens to use the Average Periodogram Analysis 
method for doing its job.

\subsection{Average periodogram analysis}
This method applies the \gls{fft}  (fast fourier transform) algorithm to the 
estimation of power spectra by sectioning the input time-domain data, finding
the modified periodogram for each section and then averaging them to get the
final spectral estimate. Since this method transforms the input data into 
smaller chunks, it requires less storage memory to implement this algorithm 
and also fewer computations. This is certainly an advantage and makes this 
method work faster. This method is also good for nonstationarity tests because
it has the potential to resolve the data in the time domain \cite{welch67}.

While computing the modified periodogram the data is smoothened using a window
because sharp transitions of data corrupt the modified periodograms. The 
window used in our project is the Blackman-Harris window.

Assume $X(n), n=0, ..., N-1$ is a sample from a stationary sequence. We take
segments, possibly overlapping, of length $L$ that are $D$ units apart from 
$X(n)$. Let $X_0(n)$ be the first such segment. Then $X_1(n) = X_0(n+D)$ and
$X_2(n) = X_1(n+D)$ and so on, where $n=0, ..., L-1$. We take $K$ segments
such that those $K$ segments almost use up all the data in $X(n)$ i.e.
$ (K-1)D + L = N$. Supposing $W(n), n=0, ..., L-1$ is the window, we calculate
the \gls{dft} of each segment to get
\begin{equation*}
    A_k(m) = \frac{1}{L}\sum_{n=0}^{L-1}X_k(n)W(n)e^{-j2\pi nm/L} \qquad
    k = 0, ..., K-1
\end{equation*}

The $K$ modified periodograms are computed as 
\begin{equation*}
    I_k(f_m) = \frac{L}{U}\left|A_k(m)\right|^2 \qquad k = 0, ..., K-1
\end{equation*}
where
\begin{equation*}
    f_m = \frac{m}{L} \qquad m = 0, ..., L/2
\end{equation*}
and
\begin{equation*}
    U = \frac{1}{L}\sum_{n=0}^{L-1}W^2(n)
\end{equation*}
The spectral estimate is the average of these periodograms,
\begin{equation*}
    P(f_m) = \frac{1}{K}\sum_{k=0}^{K-1}I_k(f_m)
\end{equation*}

\subsection{Wideband Spectrum Analyzer}

GNURadio comes with a default spectrum analyzing program named 
``usrp\_spectrum\_sense.py''. This program can be used as a template for other
programs that involve spectrum analysis.

The output of this code is the squared magnitude of the \gls{fft}  spectrum. For a 
typical \gls{fft}  bin [i] the output is 
$Y[i] = re[X[i]]*re[X[i]] + im[X[i]]*im[X[i]]$. The power for a particular 
band can be calculated by summing these values for the correct number of bins
that cover that bandwidth. The energy in a particualar frequency corresponding
to a particular bin [i] is the square root of $Y[i]$. 

$N$ time samples of $x(t)$ sampled at a sampling frequency of $F_{s}$ are 
required to use $N$ point complex \gls{fft}  $X(\omega)$ analysis. To reduce spectral
leakage, an appropriate window function like Blackman-Harris window is to be 
chosen  and applied to these time samples. The output of the \gls{fft}  represents 
the spectrum content as follows:

\begin{itemize}
    \item The first value of the \gls{fft}  output $(bin0 == X[0])$ corresponds to 
    the energy in the centre frequency.
    \item The first half of the \gls{fft}  spectrum ($X[1]$ to $X[N/2-1]$) 
    corresponds to the positive baseband frequencies i.e. from the centre 
    frequency to $+F_{s}/2$.
    \item The second half ($X[N/2]$ to $X[N-1]$) corresponds to the negative 
    baseband frequencies, i.e. from $-F_{s}/2$ to the centre frequency.
\end{itemize}


For the purposes of our project, we used to collect N = 1024 samples using a 
radio tuner centered at the frequency of our interest, say 900 MHz. The number
of \gls{fft}  points, N should be a power of 2 for the \gls{fft}  algorithm to work fast, so
we set N = 1024 for our work. We set the sampling frequency to be 1 MHz 
because all we are required to do is scan \gls{gsm} bands that are of 200 KHz each. 
The frequency resolution thus turns out to be: 1 MHz / 1024 = 976.56 KHz. The 
decimation factor is defined as the dsp rate divided by the sampling rate. The
dsp rate is the actual hardware sampling rate of the ADC in the \gls{usrp}  kit and 
was 100 MHz in our case. The \gls{uhd}  driver (driver software for the \gls{usrp}  kits)
requires the decimation factor to be an integer value. So we set our sampling 
rate to 1 MHz which made the decimation factor to be 100 MHz / 1 MHz = 100, an
integer value. For example if we set the sampling rate as 3 MHz then the 
\gls{usrp} 's driver would complain because the decimation factor 100 MHz / 1 MHz = 
33.33 turns out to be a non-integral value.

As has been said earlier, a \gls{gsm} band is of 200 KHz. Hence, to calculate the 
energy in a particular \gls{gsm} band of interest, we have to find the average of 
all the bin values which lie in the 200 KHz band centered at that frequency.


