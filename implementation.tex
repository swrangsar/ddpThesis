\chapter{Implementation of a cognitive Base Transceiver Station in GSM band 
using OpenBTS and spectrum sensing techniques}

In this project we try to demonstrate a more efficient 
way of utilizing the spectral resources by having the secondary users
make use of the spectrum holes. The spectrum holes are the frequency channels
that have been licensed to the primary users but are not being used at that
particular space and time. This allows secondary users to make use of already 
licensed frequency bands instead of having to allot them completely new 
frequency bands altogether.

In the first phase of the project, we implemented a two-frequency system where
the secondary system had an option of switching into one of two frequency 
channels depending on which one was free. We expanded this to a four-frequency
system with two primary systems in the second phase. The secondary would 
search for an unused frequency band among these four frequencies, two of which
always remain used.

\section{The two-frequency system}
\subsection{Experimental setup}

Experimental setup diagram.

The hardware and software components used in this experiment are the 
following:
\begin{itemize}
    \item \textbf{A primary BTS --} This is a Linux laptop running OpenBTS
    software with 1 USRP as the OpenBTS radio interface. The USRP hardware kit
    has a WBX 50-2200 MHz RX/TX daughterboard in it. Two mobile phones 
    (primary users) are connected to the OpenBTS network running in this 
    primary BTS system.
    \item \textbf{A secondary BTS --} This is an Ubuntu desktop running
    OpenBTS and GNURadio software. Two USRP kits are connected to this
    machine, one as the OpenBTS radio interface and the other as the GNURadio
    radio interface. The GNURadio software is used for the spectrum sensing. 
    So, here the OpenBTS software with its radio interface acts as a Base
    Transceiver Station (BTS) while the GNURadio software alongwith its radio
    interface acts as a spectrum sensor. Each of the two USRP kits has a
    WBX 50-2200 MHz RX/TX daughterboard. Two other mobile phones (secondary 
    users) are connected to the OpenBTS network running in this secondary BTS.
\end{itemize}

The secondary BTS system has cognitive capabilities. It was a challenge to 
make OpenBTS and GNURadio run simultaneously in the same computer and make 
them communicate with each other. GNURadio keeps sensing the spectrum used by
the secondary users continuously in the background and takes decisions whether
to switch the frequency band of the secondary BTS or not, depending upon the 
energy level in the frequency band in which it is running.

\subsection{Testing}
First we choose any two GSM frequency bands say 945 MHz ($F_1$) and 950 MHz 
($F_2$). The primary users are made to occupy $F_1$. Then we let the secondary
users come into $F_1$. This makes the energy level in $F_1$ go high, which 
gets detected by the spectrum sensor of the secondary BTS. So, the secondary 
BTS moves out of $F_1$ and switches its frequency to $F_2$. Similarly, now if 
the primary users are made to come into $F_2$, the secondary switches back to 
$F_1$.

In this experiment we don't have the situation where both $F_1$ and $F_2$ 
remain occupied because there is only one set of primary users. Therefore, the 
secondary also doesn't check the energy level in a channel before taking the
decision to switch into that channel.

Flow graph here.


\section{The four-frequency system}

As has been said earlier, in second phase, we expanded the two-frequency
system to a four-frequency one. The frequency channels are $F_1$ = 936 MHz, 
$F_2$ = 943 MHz, $F_3$ = 950 MHz, $F_4$ = 957 MHz. We also had two primary 
systems instead of just one this time. We also used a method known as CUSUM 
for peak detection in this case.
\subsection{Experimental setup}
The tools used in this experiment are as follows:
\begin{itemize}
    \item \textbf{Two primary BTSs --} One is a laptop and the other one is a
    desktop. Both of them runs Ubuntu as the Operating System. Each one of
    them runs OpenBTS with a USRP kit as its radio interface. A pair of mobile
    phones are connected to each one of them.
    \item \textbf{A secondary BTS --} This is the same as in the 
    two-frequency system. It runs OpenBTS and GNURadio on two different USRP
    kits. A pair of mobile phones (secondary users) are connected to its
    OpenBTS network.
\end{itemize}

One of the primary BTSs has a USRP with a SBX 400-4400 MHz RX/TX 
daughterboard, the rest of the USRPs all had a WBX daughterboard as before.

\subsection{Testing}

Initially we make one of the primary systems operate in $F_2$. And the 
secondary is made to operate in $F_1$. Now we let the other primary come into 
$F_1$. The secondary senses it and attempts to switch to $F_2$ because the 
secondary is programmed to check $F_1$, $F_2$, $F_3$, $F_4$ serially in that 
order. After checking $F_4$ the secondary checks $F_1$, $F_2$, $F_3$, ... 
again and so on the cycle continues.

But the frequency $F_2$ happens to be occupied by the one of the primary 
systems. So, the secondary moves ahead to $F_3$ which is unoccupied and 
utilizes that channel.

Unlike the two-frequency system, in this case the secondary always checks 
the availability of a channel before deciding to switch into it. In the 
two-frequency system, it was assumed that one of the two channels is always
unoccupied.

Flow graph for four-frequency system here.

The spectrum sensing is done using the energy detection based method. For 
making decisions whether to switch the frequency of the secondary BTS or not, 
a threshold of energy level is required to be set. If the energy level in a 
given channel is beyond the thresho ld, it implies that the primary users are
also using that channel i.e. that channel is already occupied by primary 
users.

To choose a threshold energy level, we took various readings of the noise 
floor, energy when only primary users were active, energy when only secondary 
users were active and also when both primary users and secondary users were 
active simultaneously for a short duration. The energy level threshold depends
on the distance of the mobile phones from the BTS. We can subdue this 
dependency on the distance by setting the threshold quite low so that even if
the users move farther away the decision making is not affected.

\section{CUSUM method}

The CUSUM (or cumulative sum control chart) method is used here to detect 
the peak in energy levels. This method is used to ascertain that the peak in 
the energy levels in a given channel are not just due to some irrelevant 
reasons like random fluctuations in the noise power, etc.
 
CUSUM is a technique used for monitoring change detection. It involves 
calculating the cumulative sum, which is what makes it sequential. The samples
from a process $x_n$ are assigned weights $\omega_n$ and summed in the 
following way:
\begin{align}
    S_0 &= 0 \nonumber \\
    S_n &= max(0, S_n + x_n - \omega_n) \nonumber
\end{align}

When the value of $S$ exceeds a certain threshold value, a change in value has 
been found. However, this formula detects only a change in the positive 
direction. For the negative direction, the $max$ operation has to be replaced 
by the $min$ operation. In this case the change has been found when the 
(negative) value is below the threshold value.





