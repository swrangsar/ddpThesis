\chapter{Codes}

\section{secondaryBTS.py}

This code was written by modifying the already available program named 
``usrp\_spectrum\_sense.py'' that comes together with the GNURadio software
package. We set the default UHD device address to ``192.168.20.2'' because 
that is the IP address of the USRP device we are using as a spectrum sensor.
The default samping rate was set to 1e6 i.e. 1MHz. The default FFT size is
given as 'sampling rate / channel bandwidth'. We wanted an FFT size of 1024 so
we set the bandwidth to 976.56 Hz since 1 MHz / 976.56 Hz $\approx$ 1024.

The class 'my\_top\_block' was modified by replacing the lines:
\begin{lstlisting}[language=Python]

        self.channel_bandwidth = options.channel_bandwidth

        self.min_freq = eng_notation.str_to_num(args[0])
        self.max_freq = eng_notation.str_to_num(args[1])

        if self.min_freq > self.max_freq:
            # swap them
            self.min_freq, self.max_freq = self.max_freq, self.min_freq    
\end{lstlisting}
with the lines
\begin{lstlisting}[language=Python]
        self.channel_bandwidth = options.channel_bandwidth

        self.down_freq = eng_notation.str_to_num(args[0])
        self.up_freq = (self.down_freq) - 45e6    
\end{lstlisting}

The method 'set\_next\_freq' of the class 'my\_top\_block' was modified by
replacing
\begin{lstlisting}[language=Python]
        target_freq = self.next_freq
        self.next_freq = self.next_freq + self.freq_step
        if self.next_freq >= self.max_center_freq:
            self.next_freq = self.min_center_freq
\end{lstlisting}
with
\begin{lstlisting}[language=Python]
        target_freq = self.up_freq
\end{lstlisting}




In the code listing that follows we have listed only the functions that we
customized and the ones that we added.

\begin{lstlisting}[language=Python]

def main_loop(tb):
    startOpenBTS(tb.down_freq,tb)


def sub_loop(tb):

    print 'fft size', tb.fft_size
    N = tb.fft_size
    mid = N // 2
    cusum = 0
    counter = 0
    

    while 1:

        # Get the next message sent from the C++ code (blocking call).
        # It contains the center frequency and the mag squared of the fft
        m = parse_msg(tb.msgq.delete_head())

        # m.center_freq is the center frequency at the time of capture
        # m.data are the mag_squared of the fft output
        # m.raw_data is a string that contains the binary floats.
        # You could write this as binary to a file.



        center_freq = m.center_freq
        bins = 102
        power_data = 0
        noise_floor_db = 0      ##  10*math.log10(min(m.data)/tb.usrp_rate)
        
        for i in range(1, bins+1):
            power_data += m.data[mid-i] + m.data[mid+i]
        power_data += m.data[mid]
        power_data /= ((2*bins) + 1)
        
        power_db = 10*math.log10(power_data/tb.usrp_rate) - noise_floor_db
        power_threshold = -70.0
        
        print datetime.now(), "center_freq", center_freq, "power_db", power_db

        #cusum cusum cusum is here
        cusum = max(0, cusum + power_db - power_threshold)
        if (cusum > 0):
            counter += 1
            if (counter > 2):
                print "CUSUM is now positive!!!"
                down_freq = center_freq + 45e6
                quitOpenBTS(down_freq, tb)
                break
        else:
            counter = 0




def startOpenBTS(downFrequency,tb):
    arfcn=int((downFrequency-935e6)/2e5)
    if (arfcn < 0):
        print "ARFCN must be > 0 !!!"
        sys.exit(1)
    print 'ARFCN=', arfcn
    #DB modifications
    t=(arfcn,)
    conn=sqlite3.connect("/etc/OpenBTS/OpenBTS.db")
    cursor=conn.cursor()
    cursor.execute("update config set valuestring=? where keystring='GSM.Radio.C0'",t)
    conn.commit()

    #start the OpenBTS
    f=subprocess.Popen(os.path.expanduser('~/ddpOpenBTS/runOpenBTS.sh'))
    f.wait()
    tb.msgq.delete_head()
    time.sleep(0.25)
    sub_loop(tb)
              

def quitOpenBTS(downFreq, tb):
    f=subprocess.Popen(os.path.expanduser('~/ddpOpenBTS/quitOpenBTS.sh'))
    f.wait()    
    newDownFreq = getNewChannel(downFreq, tb)
    startOpenBTS(newDownFreq, tb)

        

def getNewChannel(downFreq, tb):
    newDownFreq = downFreq + 7e6
    if newDownFreq > 960e6:
        newDownFreq = 936e6

    tb.up_freq = newDownFreq - 45e6
    print "new tb.up_freq: ", tb.up_freq
    tb.msgq.delete_head()
    time.sleep(0.25)

    print 'fft size', tb.fft_size
    N = tb.fft_size
    mid = N // 2
    cusum = 0
    counter = 0
    loopcounter = 0
    

    while loopcounter < 10:

        # Get the next message sent from the C++ code (blocking call).
        # It contains the center frequency and the mag squared of the fft
        m = parse_msg(tb.msgq.delete_head())


        center_freq = m.center_freq
        bins = 102
        power_data = 0

        
        for i in range(1, bins+1):
            power_data += m.data[mid-i] + m.data[mid+i]
        power_data += m.data[mid]
        power_data /= ((2*bins) + 1)
        
        power_db = 10*math.log10(power_data/tb.usrp_rate)
        power_threshold = -70.0
        
        
        print datetime.now(), "center_freq", center_freq, "power_db", power_db
        print "precheck"

        #cusum cusum cusum is here
        cusum = max(0, cusum + power_db - power_threshold)
        loopcounter += 1
        if (cusum > 0):
            counter += 1
            if (counter > 2):
                print "CUSUM is now positive!!!"
                newDownFreq = getNewChannel(newDownFreq, tb)
                break
        else:
            counter = 0
    return newDownFreq

\end{lstlisting}




\section{primaryBTS.py}
\begin{lstlisting}[language=Python]
#!/usr/bin/env python


import sys
import sqlite3
import os
import re

def main_loop():
    usage = "usage: %prog channel_freq"
    if len(sys.argv) != 2:
        print 'usage:', sys.argv[0], 'channel_freq'
        sys.exit(1)


    center_freq = int(re.match(r'\d+', sys.argv[1]).group())*1e6
    startOpenBTS(center_freq)

def startOpenBTS(frequency):            
    
    arfcn=int((frequency-935e6)/2e5)
    print 'ARFCN=', arfcn
    
    #DB modifications
    t=(arfcn,)
    conn=sqlite3.connect("/etc/OpenBTS/OpenBTS.db")
    cursor=conn.cursor()
    cursor.execute("update config set valuestring=? where keystring='GSM.Radio.C0'",t)
    conn.commit()
    
    #start the OpenBTS
    f=os.popen('~/ddpOpenBTS/runOpenBTS.sh')
    f.close()
              

if __name__ == '__main__':

    try:
        main_loop()

    except KeyboardInterrupt:
        pass

\end{lstlisting}


\section{runOpenBTS.sh}
\begin{lstlisting}[language=bash]
#!/bin/bash

sudo echo "Hi, this script starts OpenBTS in Ubuntu 12.04!"
sudo service asterisk restart
sudo gnome-terminal -x sh -c "sudo asterisk -r" &

#cd ~/OpenBTS/
#sudo gnome-terminal --tab -e "sudo smqueue/trunk/smqueue/smqueue" \
#   --tab -e "sudo subscriberRegistry/trunk/sipauthserve" &

cd ~/OpenBTS/openbts/trunk/apps/
sudo gnome-terminal --tab -e "sudo ../../../smqueue/trunk/smqueue/smqueue" \
    --tab -e "sudo ../../../subscriberRegistry/trunk/sipauthserve" &

#sudo gnome-terminal -x sh -c "sudo ./OpenBTS" &
#sudo gnome-terminal -x sh -c "sudo ./OpenBTSCLI" &
sudo gnome-terminal --tab -e "sudo ./OpenBTS" \
    --tab -e "sudo ./OpenBTSCLI" &
cd ~

\end{lstlisting}


\section{quitOpenBTS.sh}
\begin{lstlisting}[language=bash]
#!/bin/bash

sudo echo "Hi, this script turns OpenBTS off in Ubuntu 12.04!"

sudo killall transceiver smqueue sipauthserve OpenBTSCLI asterisk
\end{lstlisting}




