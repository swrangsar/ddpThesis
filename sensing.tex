\chapter{Spectrum sensing}

Spectrum sensing is the main task in the entire operation of cognitive
radio. Spectrum sensing is defined as the finding of spectrum holes in 
the local neighborhood of the cognitive radio receiver. Spectrum holes are the
underutilized (in part or in full) subbands of spectrum at a particular time
in a specific location. Moreover for cognitive radio to fulfil its potential
in solving the problem of spectrum underutilization, the spectrum sensing 
method used should be reliable and computationally feasible in real-time 
\cite{haykin09}.

There are many spectrum sensing techniques available. Three important ones of
them are as follows:
\begin{itemize}
    \item Energy detection
    \item Matched filter detection
    \item Cyclostationarity detection
\end{itemize}

\section{Energy detection}

Conventional energy detector is made up of a low pass filter, an A/D converter
, a square law detector and an integrator. This implementation is not flexible
enough, especially in the case of narrowband signals and sinewaves. Also, this
requires a pre-filter matched to the bandwidth $B$ of the signal to be 
scanned.

So, an alternative implementation is generally used where we find the squared 
magnitude of the FFT using the Average Periodogram method. In this 
architecture, we can alter the bandwidth of frequencies scanned just by taking
the required number of FFT bins. 

Spectrum sensing can be viewed as a binary hypothesis-testing problem 
\cite{zhang09}:
\begin{itemize}
    \item $H0$: primary user is absent
    \item $H1$: primary user is present
\end{itemize}



