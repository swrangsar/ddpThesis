\chapter*{\centering{Abstract}}
Wireless networks are currently regulated by fixed spectrum assignment policy
which results in inefficient utilization of the spectrum. In the last few
years there has been a drastic increase in the mobile services, which in turn
has increased the demand for limited radio spectrum. Hence there is a need to
change the conventional static spectrum assignment policy and make efficient
utilization of spectrum. Cognitive Radio (CR) emerged as a new paradigm to
address the spectrum underutilization problem.  CR enables opportunistic use
of the radio-frequency spectrum by allowing Secondary/Un-licensed users to
utilize licensed bands under the condition that they should not cause the
interference with the Primary/Licensed users. Secondary users utilize the
detected free bands and leave them when the corresponding primary radio
emerges.

A USRP based two frequency CR testbed and a four frequency CR testbed has been
developed using GNURadio and OpenBTS to demonstrate these properties of
cognitive radio. Primary and secondary users are made to coexist with no
effect on the primary radio. Energy detection spectrum sensing technique is
used to detect the free bands also known as the spectrum holes. We demonstrate
that secondary users utilize these spectrum holes and vacate them whenever
primary users want to use them.
