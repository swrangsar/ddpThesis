\chapter{Conclusion and future work}

\section{Conclusion}
Cognitive radio is the solution to current day problem of inefficient 
utilization of the radio spectrum. It identifies the spectrum holes and 
enables secondary users to utilize these holes for communication thereby 
increasing total number of mobile users. We have demonstrated these cognitive
radio capabilities using a 2-frequency and a 4-frequency cognitive radio
test-bed and successfully testing them.  We demonstrated that the secondary 
users quit the frequency channel which they are utilizing for communication
as soon as the corresponding primary radio emerges thereby giving higher
priority to the primary users.


\section{Future work}
We used energy based spectrum sensing technique for the detection of free 
channels to be utilized by secondary users. The algorithm we have designed has
dependency on the distance between the base station and mobile station as the
energy varies with distance. This dependency can be removed by designing a 
better algorithm in future. Also we have made secondary users to vacate the
channel as soon as they corresponding primary radio appears causing the
ongoing secondary call to drop. An algorithm that can avoid this call drop and
handle this scenario in a better way can be designed in future.