\chapter{Four-frequency Cognitive Radio Test-Bed}

The 2-frequency system is expanded to a 4-frequency system by adding 
another primary subsystem to the 2-frequency experimental setup. Here we 
assume radio spectrum to be made up of four freqencies
instead of two. Secondray subsystem finds optimum channel out of 
these four frequencies for openBTS operation.
\begin{figure}
\centering
\includegraphics[width=1\textwidth]{../images/freq4}
\caption[Experimental setup, 4-frequency system]{Experimental setup of the 
4-frequency system}
\label{freq4}
\end{figure}

\subsection{4-frequency system description}
 The experimental setup of the four freqency system has 
has two primary subsystems and one secondary subsystem as described
in figure \ref{freq4}. Primary subsystem 
has OpenBTS and one USRP kit and secondary subsystem has OpenBTS and GNU radio 
software and two USRP kits as we had previously in 2-frequency system. 
Here we have four GSM bands with centre frequencies 
$F_1$=936MHz, $F_2$=943MHz, $F_3$=950MHz, $F_4$=957MHz as the radio spectrum.
 



\subsection{4-frequency system testing}
Here we make one pair of primary users occupy one of the four frequencies say 
$F_2$. We make secondary users use one frequency say $F_1$. Now other pair of 
primary users are made to try and enter frequency $F_1$ for communication. This is sensed by 
the secondary system and it tries to migrate secondary users to $F_2$ which is 
also occupied. Our secondary system detects that $F_2$ is occupied and therefore 
continues to find a spectrum hole in a 4-frequency spectrum. It finds that 
frequency $F_3$ is unoccupied and thus allows secondary users to enter $F_3$ 
and utilize it for communication. The difference between a 4-frequency 
system and a 2-frequency system is that in a 4-frequency 
system the secondary subsystem has to first confirm absence of primary users 
in the band before making secondary users migrate to that band . This was not
the case in 2-frequency system as it has only one 
pair of primary users. Hence the other band where secondary users will be made 
to migrate is always unoccupied at the time of switching secondary radio to 
that band. So there is no need to check for the peresence of primary users in that band.

The flow graph in figure \ref{freqSys4} describes the algorithm for secondary subsystem of the 4-frequency cognitive system.
 We begin with optimal channel selection out of four frequencies 936MHz, 943MHz, 950MHz, 957MHz by using peridogram analysis
. OpenBTS is started on this choosen frequency. Then we continously check for the presence of primary radio on this channel
 after every time interval 't'. If energy goes above predefined threshold OpenBTS of the secondary subsystem is switched
 off and restarted on a frequency which is seven more than previous frequency. If the previous frequency of operation was 
957MHz than the new frequency is 936MHz. From here we go back to the second step of algorithm that is spectrum scanning. If there are issues in switching off of OpenBTS when algorithm reaches end.

\begin{figure}
\centering
\includegraphics[width=\textwidth]{../images/freqSys4}
\caption[4-frequency system]{Flowchart for 4-frequency system}
\label{freqSys4}
\end{figure}

The spectrum sensing is done by energy detection technique and it was required 
that a proper threshold is set for decision making. A number of readings were 
taken to decide the noise level, energy level when only primary users are 
occupying and also energy levels when both primary users and secondary users 
exist in the same band for a short duration of time. The threshold value 
depends on the power transmitted by the users and their distance from the USRP 
kits which is RF front for GNU radio. This distance dependency can be removed by 
setting the threshold much lower so that even if the users move 
far away, the decision making is not affected.  






\section{Achievements}
The next section of this chapter gives a step by step description of the tasks we 
accomplished during our project tenure. 

\begin{enumerate}

\item We began by exploring what cognitive radio is and how it can be used 
in the already existing radio. We did a literature survey on the ongoing work in the
field of cognitive radio.

\item We learned how to use the GNURadio software package starting from its installation procedure. We 
also designed an FM receiver using GNURadio Companion to get used to the software. 
Also we tried and learnt the codes of already existing signal processing blocks 
that GNU Radio provides.

\item GNURadio applications are primarily written in the Python programming 
language and hence we learned the language, Python.

\item USRP N210 kit is used as hardware in our project. We got used to this kit and 
also carried out range testing of this kit to ensure distance is not a major factor in 
our decision making algorithm and can be neglected.
\item The next task was to understand the working of OpenBTS software. Starting 
with the installation of this software we registered our GSM SIM cards in the 
local network established by OpenBTS. We could perform calling and sending SMS
between our phones using the local network established by OpenBTS with USRP 
kit as its radio interface.
\item Since spectrum sensing is major part of cognitive radio, literature survey 
on various spectrum sensing techniques was done. We chose energy detection 
spectrum sensing technique for our project and so we did detailed study of a 
technique called Average Periodogram Analysis to implement this method. We also 
simulated this technique in Matlab using various windowing methods and 
understand results. 
\item After all this the problem statement was designed and a flow graph of 
how this problem will be approached was constructed. We also decided upon the 
experimental setup required for this problem. Detailed discription of all this
is included in the previous sections of this chapter. 
 
\item  First key step to approach the problem was to run Open BTS and GNU 
Radio together in the same computer with two USRP kits connected one for 
OpenBTS and the other for GNURadio. It was a little tricky because we had to
find out whether it was possible to run two USRP kits on the same computer simultaneously.
Fortunately, it turned out it was possible if the two kits had two different IP addresses.
We then tried to figure out how to burn a different IP address to a USRP kit and
managed to do the same.
\item The next step was to build a 2-frequency cognitive system with a pair of 
primary and secondary users communicating in parallel and primary pair having 
higher priority when ever both try to use same radio band. Detailed description 
is in the previous sections of this chapter.
\item This 2-frequency system was expanded to 4-frequency system with two 
primary pair of users and one secondary pair of users coexisting. This 
demonstrated that both primary and secondary users can exist in the same GSM 
Network without affecting the existing system.
\end{enumerate}