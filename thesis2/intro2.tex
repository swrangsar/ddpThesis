\chapter{Introduction}

\section{Motivation}
Due to the rapid increase of mobile phones and other wireless communication devices, 
there is a need for efficient utilization of the available radio spectrum.
The Spectrum Policy Task Force, a group under the Federal Communications  
Commission (FCC) in the United States, published a report in 2002 saying \cite{repFCC}: 
\begin{quote}
``In many bands, spectrum access is a more significant problem than physical
scarcity of spectrum, in large part due to legacy command-and-control regulation that limits the ability of potential spectrum users to obtain such access.''
\end{quote}
If we scan the spectrum in metropolitan cities which are heavily used regions, we find that some frequency bands are unoccupied most of the time[7]. These are referred to as spectrum holes. A spectrum hole is a band of frequencies assigned to a primary user, but, at a particular time and specific geographic location, the band is not being utilized by that user.[4]


Diagram of spectrum here 

This problem of inefficient utilization of spectrum can be solved by allowing secondary users which are non licensed, to access these spectrum holes. Cognitive radio which includes software defined radio, is a means to accomplish this by utilizing these spectrum holes intelligently and efficiently.5,2,6 kranti. It uses one of the spectrum sensing techniques to identify the spectrum holes in the radio spectrum.


\section{Cognitive Radio}
A cognitive radio is an intelligent radio whose primary objective is efficient utilization of the radio spectrum. It can be programmed and configured dynamically. It works on the principle of understanding-by-building to learn from the surrounding environment and adapt to changes in the RF stimuli by making corresponding changes in operating parameters. The transceiver is designed to find an unoccupied channel in the vicinity and utilize it for transmission. It enables coexistence of primary licensed users and secondary unlicensed users. Whenever a primary user wants to occupy the channel which is currently in use by secondary users, it finds some other unoccupied channel in the vicinity and secondary users migrate seamlessly to this new channel thus vacating the previously used channel for primary users.  

\section{Contribution of thesis}
An experimental setup is developed which demonstrates the presence of secondary users along with primary users in the existing GSM network and utilizing the already existing resources there by increasing the total mobiles in the network.
\begin{enumerate}
	\item A two frequency band cognitive system is developed where secondary users migrate to frequency f2 if frequency f1 is occupied an viceverca.
	\item A two frequency system is extended to a four frequency system where we demonstrate that primary users are occupying two bands out of these four and secondary users occupy one out of the other two free bands.
	\item  We have used energy detection spectrum sensing technique and CUSUM peak detection technique to detect the presence of primary users. Band occupied by secondary users is continuously monitored to check if primary users are trying to occupy that band and as soon as the request from primary users is detected a new free band is found out in the vicinity and utilized by secondary users for transmission there by vacating the band for primary users . 
\end{enumerate}

\section{ORGANIZATION}
The rest of this document is organized as follows. Chapter 2 briefly describes the GSM architecture and its Um interface. Chapter 3 gives a literature survey on Universal Software Radio Peripheral (USRP N210) the hardware used in this project. Literature survey done on the GNU Radio software package and OpenBTS software is described in Chapter 4 and 5 respectively. Chapter 6 covers spectrum sensing techniques to detect the presence of primary users in the channel.  Chapter 7 covers a implementation of cognitive radio using GNU Radio and OpenBTS. It describes the experimental setup for our project in the beginning followed by detailed description of what we have achieved in this project along with a flow chart of our work. The final chapter of this thesis is the conclusion of our project followed by future work. 
