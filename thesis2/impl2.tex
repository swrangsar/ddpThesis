\chapter{Implementation of cognitive radio using OpenBTS}

In our project we are able to successfully demonstrate the
coexistence of primary users and secondary users in the same 
frequency channel in the
GSM band. In order to accomplish this, we have implemented
a cognitive radio which detects the spectrum holes in the
radio spectrum and enables secondary users to utilize these 
for communication. An experimental setup has been  
developed for this demonstration using OpenBTS and GNU 
radio software and USRP N210 as hardware.

Experimental setup diagram:


\section{Description of setup}
The figure above describes the experimental setup for a
two-frequency system. The primary system has only
one USRP as an RF front and it runs OpenBTS. The 
secondary system has two USRP kits connected to it and one of them
runs OpenBTS and the other GNURadio.
This secondary system has cognitive capabilities. 
To provide cognitive capabilities it was required that 
OpenBTS and GNU radio run together in the same system and 
talk to each other which was challenging. Secondary system
continuously senses the frequency band of interest and does 
decision making depending upon the analysis of the data 
collected and changes its parameters accordingly so that 
primary and secondary users coexist. The spectrum sensing 
is accomplished by using GNU radio. Also we made GNU radio 
and OpenBTS coordinate to behave in appropriate manner and 
take dynamic decisions as and when required to make over all
system behave in a cognitive manner.

First a two-frequency cognitive system is developed.
For this two GSM bands are used with centre frequency
945 MHz ($F_1$) and 950 MHz ($F_2$). Secondary users are made to occupy
one of these two bands say $F_1$. Then we make the primary users 
enter the same band. This results in an increase in energy 
levels in this band which is sensed by the secondary system 
as it continuously scans this band. Immediately
secondary users are shifted to other frequency band ($F_2$)
there by vacating $F_1$ for primary users. Hence a two-frequency
cognitive system demonstrating coexistence of a pair of
primary and secondary users is accomplished. 

The whole technique is described using a flow graph below:

Flow graph here :spectrumSensing2

Now this two frequency system is expanded to a 
four frequency system where we have $F_1$ = 936 MHz, 
$F_2$ = 943 MHz, $F_3$ = 950 MHz, $F_4$ = 957 MHz. The experimental setup
is also expanded with two primary systems and one secondary
system. Each primary system has an USRP kit running OpenBTS and
the secondary system has two USRP kits connected to it, one for OpenBTS
and the other one for GNURadio, as we had previously in the two-frequency 
system.
 
Figure for 4 frequency system here:


Here we make a pair of primary users occupy one of the four
frequency channel say $F_2$. We make secondary users use a frequency channel
say $F_1$. Now the other pair of primary users try to enter frequency
$F_1$ for communication. This is sensed by the secondary system 
and it tries to migrate secondary users to $F_2$ which also happens to be 
occupied already. Our secondary system detects that $F_2$ is occupied and
therefore moves on to find a spectrum hole in the four-frequency 
spectrum. It finds that frequency $F_3$ is unoccupied and thus
allows secondary users to enter $F_3$ and utilize it for communication.
The difference between a four-frequency system and a two-frequency system is 
that the secondary system in a four-frequency
system has to first check the presence of primary users before 
switching into a particular frequency channel.
This was not the case in the two-frequency system. In the
two-frequency system we assumed that the other band is always 
unoccupied at the time of switching as only a pair of primary
users existed and thereby only a single band is always unoccupied.


The following flow graph describes the four frequency cognitive system:

Flow graph here

The spectrum sensing is done by energy detection technique and
it was required that a proper threshold be set for decision making. 
A number of readings were taken to decide the noise level, energy 
level when only primary users were active and also energy levels 
when both primary users and secondary users were active in the same band 
for a short duration of time. The threshold value depends on the
power transmitted by the users and their distance from the USRP 
kit which is RF front for GNURadio. This distance dependency can 
be removed by setting the threshold quite lower than required so 
that even if the users move far away the decision making is not
affected. 






\section{CUSUM}
CUMSUM peak detection is also applied after energy detection to ensure that the
detected high energy in the band of interest is not due some  not relevant 
reason like random fluctuations in noise power etc but due to presence of 
primary radio in that band. This ensures high accuracy in primary radio 
detection and correct decision making.
CUSUM is basically a sequential analysis technique to monitor change detection. 
It is a criterion for deciding when to take corrective action. As the name 
implies CUSUM involves calculating cumulative sum. This makes it sequential. 
The samples from a process $x_n$  are assigned weights $\omega_n$  and summed 
as followed,

\begin{align}
S_0 &= 0; \nonumber \\
S_{n+1} &= max(0, S_n + x_n - \omega_n) \nonumber
\end{align}
When the value of $S$ exceeds a certain threshold value a change in value has 
been found. This formula detects change only in positive direction. To detect 
change in negative direction we have to do min operation instead of max 
operation and the change in value is detected when goes below threshold.







\section{Tasks undertaken over the year}
The next section of this chapter gives a step by step description of the tasks we 
accomplished during our project tenure. 

\begin{enumerate}

\item We began by exploring what cognitive radio is and how it can be used 
in the already existing radio. We did a literature survey on the ongoing work in the
context of cognitive radio.

\item We learned how to use the GNURadio software package starting from its installation procedure. We 
also designed an FM receiver using GNURadio Companion to get used to the software. 
Also we tried and learnt the codes of already existing signal processing blocks 
that GNU Radio provides.

\item GNURadio applications are primarily written in the Python programming 
language and hence we learned the language, Python.

\item USRP N210 kit is used as hardware in our project. We got used to this kit and 
also performed range testing of this kit to ensure distance is not a major factor in 
our decision making algorithm and can be neglected.
\item The next task was to understand the working of OpenBTS software. Starting 
with the installation of this software we registered our GSM SIM cards in the 
local network established by OpenBTS. We could perform calling and sending SMS
between our phones using the local network established by OpenBTS with USRP 
kit as its radio interface.
\item Since spectrum sensing is major part of cognitive radio literature survey 
on various spectrum sensing techniques was done. We choose energy detection 
spectrum sensing technique for our project and so we did detail study of a 
technique called Average Periodogram Analysis to implement this method. We also 
simulated this technique in Matlab using various windowing methods and 
understand results. 
\item After all this we designed our problem statement and the flow graph of 
how we will approach the problem which we have included in the last chapter. 
Also we decided upon building the experimental setup which we have covered in 
the start of this chapter. 
\item  First key step to approach this problem was to run Open BTS and GNU 
Radio together in the same computer with two USRP kits connected one for 
OpenBTS and the other for GNURadio. It was a little tricky because we had to
find out whether it was possible run two USRP kits on the same computer simultaneously.
Fortunately, it turned out it was possible if the two kits had two different IP addresses.
We then tried to figure out how to burn a different IP address to a USRP kit and
managed to do the same.
\item The next step was to build a two frequency cognitive system with a pair of 
primary and secondary users communicating in parallel and primary pair having 
higher priority when ever both try to use same radio band. Detailed description 
is in the start of this chapter.
\item This two frequency system was expanded to four frequency system with two 
primary pair of users and one secondary pair of users coexisting. This 
demonstrated that both primary and secondary users can exist in the same GSM 
Network without affecting the existing system.
\end{enumerate}