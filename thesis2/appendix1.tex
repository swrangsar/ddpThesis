\chapter{Codes}

\section{secondaryBTS.py}
\begin{lstlisting}[language=Python]
#!/usr/bin/env python
#
# Copyright 2005,2007,2011 Free Software Foundation, Inc.
#
# This file is part of GNU Radio
#
# GNU Radio is free software; you can redistribute it and/or modify
# it under the terms of the GNU General Public License as published by
# the Free Software Foundation; either version 3, or (at your option)
# any later version.
#
# GNU Radio is distributed in the hope that it will be useful,
# but WITHOUT ANY WARRANTY; without even the implied warranty of
# MERCHANTABILITY or FITNESS FOR A PARTICULAR PURPOSE.  See the
# GNU General Public License for more details.
#
# You should have received a copy of the GNU General Public License
# along with GNU Radio; see the file COPYING.  If not, write to
# the Free Software Foundation, Inc., 51 Franklin Street,
# Boston, MA 02110-1301, USA.
#

from gnuradio import gr, eng_notation
from gnuradio import blocks
from gnuradio import audio
from gnuradio import filter
from gnuradio import fft
from gnuradio import uhd
from gnuradio.eng_option import eng_option
from optparse import OptionParser
import sys
import math
import struct
import threading
import time
import sqlite3
import os
import subprocess
from datetime import datetime

sys.stderr.write("Warning: this may have issues on some machines+Python version combinations to seg fault due to the callback in bin_statitics.\n\n")

class ThreadClass(threading.Thread):
    def run(self):
        return

class tune(gr.feval_dd):
    """
    This class allows C++ code to callback into python.
    """
    def __init__(self, tb):
        gr.feval_dd.__init__(self)
        self.tb = tb

    def eval(self, ignore):
        """
        This method is called from blocks.bin_statistics_f when it wants
        to change the center frequency.  This method tunes the front
        end to the new center frequency, and returns the new frequency
        as its result.
        """

        try:
            # We use this try block so that if something goes wrong
            # from here down, at least we'll have a prayer of knowing
            # what went wrong.  Without this, you get a very
            # mysterious:
            #
            #   terminate called after throwing an instance of
            #   'Swig::DirectorMethodException' Aborted
            #
            # message on stderr.  Not exactly helpful ;)

            new_freq = self.tb.set_next_freq()
            
            # wait until msgq is empty before continuing
            while(self.tb.msgq.full_p()):
                #print "msgq full, holding.."
                time.sleep(0.1)
            
            return new_freq

        except Exception, e:
            print "tune: Exception: ", e


class parse_msg(object):
    def __init__(self, msg):
        self.center_freq = msg.arg1()
        self.vlen = int(msg.arg2())
        assert(msg.length() == self.vlen * gr.sizeof_float)

        # FIXME consider using NumPy array
        t = msg.to_string()
        self.raw_data = t
        self.data = struct.unpack('%df' % (self.vlen,), t)


class my_top_block(gr.top_block):

    def __init__(self):
        gr.top_block.__init__(self)

        usage = "usage: %prog [options] down_freq"
        parser = OptionParser(option_class=eng_option, usage=usage)
        parser.add_option("-a", "--args", type="string", default="addr=192.168.20.2",
                          help="UHD device device address args [default=%default]")
        parser.add_option("", "--spec", type="string", default=None,
                      help="Subdevice of UHD device where appropriate")
        parser.add_option("-A", "--antenna", type="string", default=None,
                          help="select Rx Antenna where appropriate")
        parser.add_option("-s", "--samp-rate", type="eng_float", default=1e6,
                          help="set sample rate [default=%default]")
        parser.add_option("-g", "--gain", type="eng_float", default=None,
                          help="set gain in dB (default is midpoint)")
        parser.add_option("", "--tune-delay", type="eng_float",
                          default=0.25, metavar="SECS",
                          help="time to delay (in seconds) after changing frequency [default=%default]")
        parser.add_option("", "--dwell-delay", type="eng_float",
                          default=0.25, metavar="SECS",
                          help="time to dwell (in seconds) at a given frequency [default=%default]")
        parser.add_option("-b", "--channel-bandwidth", type="eng_float",
                          default=976.56, metavar="Hz",
                          help="channel bandwidth of fft bins in Hz [default=%default]")
        parser.add_option("-l", "--lo-offset", type="eng_float",
                          default=0, metavar="Hz",
                          help="lo_offset in Hz [default=%default]")
        parser.add_option("-q", "--squelch-threshold", type="eng_float",
                          default=None, metavar="dB",
                          help="squelch threshold in dB [default=%default]")
        parser.add_option("-F", "--fft-size", type="int", default=None,
                          help="specify number of FFT bins [default=samp_rate/channel_bw]")
        parser.add_option("", "--real-time", action="store_true", default=False,
                          help="Attempt to enable real-time scheduling")


        (options, args) = parser.parse_args()
        if len(args) != 1:
            parser.print_help()
            sys.exit(1)

        self.channel_bandwidth = options.channel_bandwidth

        self.down_freq = eng_notation.str_to_num(args[0])
        self.up_freq = (self.down_freq) - 45e6



        if not options.real_time:
            realtime = False
        else:
            # Attempt to enable realtime scheduling
            r = gr.enable_realtime_scheduling()
            if r == gr.RT_OK:
                realtime = True
            else:
                realtime = False
                print "Note: failed to enable realtime scheduling"

        # build graph
        self.u = uhd.usrp_source(device_addr=options.args,
                                 stream_args=uhd.stream_args('fc32'))

        # Set the subdevice spec
        if(options.spec):
            self.u.set_subdev_spec(options.spec, 0)

        # Set the antenna
        if(options.antenna):
            self.u.set_antenna(options.antenna, 0)
        
        self.u.set_samp_rate(options.samp_rate)

        self.usrp_rate = usrp_rate = self.u.get_samp_rate()
        
        self.lo_offset = options.lo_offset

        if options.fft_size is None:
            self.fft_size = int(self.usrp_rate/self.channel_bandwidth)
        else:
            self.fft_size = options.fft_size
        
        self.squelch_threshold = options.squelch_threshold
        
        s2v = blocks.stream_to_vector(gr.sizeof_gr_complex, self.fft_size)

        mywindow = filter.window.blackmanharris(self.fft_size)
        ffter = fft.fft_vcc(self.fft_size, True, mywindow, True)
        power = 0
        for tap in mywindow:
            power += tap*tap

        c2mag = blocks.complex_to_mag_squared(self.fft_size)


        tune_delay  = max(0, int(round(options.tune_delay * usrp_rate / self.fft_size)))  # in fft_frames
        dwell_delay = max(1, int(round(options.dwell_delay * usrp_rate / self.fft_size))) # in fft_frames

        self.msgq = gr.msg_queue(1)
        self._tune_callback = tune(self)        # hang on to this to keep it from being GC'd
        stats = blocks.bin_statistics_f(self.fft_size, self.msgq,
                                        self._tune_callback, tune_delay,
                                        dwell_delay)

        # FIXME leave out the log10 until we speed it up
    #self.connect(self.u, s2v, ffter, c2mag, log, stats)
    self.connect(self.u, s2v, ffter, c2mag, stats)

        if options.gain is None:
            # if no gain was specified, use the mid-point in dB
            g = self.u.get_gain_range()
            options.gain = float(g.start()+g.stop())/2.0

        self.set_gain(options.gain)
        print "gain =", options.gain

    def set_next_freq(self):
        target_freq = self.up_freq


        if not self.set_freq(target_freq):
            print "Failed to set frequency to", target_freq
            sys.exit(1)

        return target_freq


    def set_freq(self, target_freq):
        """
        Set the center frequency we're interested in.

        Args:
            target_freq: frequency in Hz
        @rypte: bool
        """
        
        r = self.u.set_center_freq(uhd.tune_request(target_freq, rf_freq=(target_freq + self.lo_offset),rf_freq_policy=uhd.tune_request.POLICY_MANUAL))
        if r:
            return True

        return False

    def set_gain(self, gain):
        self.u.set_gain(gain)
    


def main_loop(tb):
    startOpenBTS(tb.down_freq,tb)




def sub_loop(tb):

    # use a counter to make sure power is less than threshold
    # lowPowerCount = 0
    # lowPowerCountMax = 10
    print 'fft size', tb.fft_size
    N = tb.fft_size
    mid = N // 2
    cusum = 0
    counter = 0
    

    while 1:

        # Get the next message sent from the C++ code (blocking call).
        # It contains the center frequency and the mag squared of the fft
        m = parse_msg(tb.msgq.delete_head())

        # m.center_freq is the center frequency at the time of capture
        # m.data are the mag_squared of the fft output
        # m.raw_data is a string that contains the binary floats.
        # You could write this as binary to a file.



        center_freq = m.center_freq
        bins = 102
        power_data = 0
        noise_floor_db = 0      ##  10*math.log10(min(m.data)/tb.usrp_rate)
        
        for i in range(1, bins+1):
            power_data += m.data[mid-i] + m.data[mid+i]
        power_data += m.data[mid]
        power_data /= ((2*bins) + 1)
        
        power_db = 10*math.log10(power_data/tb.usrp_rate) - noise_floor_db
        power_threshold = -70.0
        

        #if (power_db > tb.squelch_threshold) and (power_db > power_threshold):
            #print datetime.now(), "center_freq", center_freq, "power_db", power_db, "in use"
            # lowPowerCount = 0
        #else:
        print datetime.now(), "center_freq", center_freq, "power_db", power_db
            # lowPowerCount += 1

        #    if (lowPowerCount > lowPowerCountMax):
        #        down_freq = center_freq + 45e6
        #        startOpenBTS(down_freq)
        #        break

        #cusum cusum cusum is here
        cusum = max(0, cusum + power_db - power_threshold)
        if (cusum > 0):
            counter += 1
            if (counter > 2):
                print "CUSUM is now positive!!!"
                down_freq = center_freq + 45e6
                quitOpenBTS(down_freq, tb)
                break
        else:
            counter = 0




def startOpenBTS(downFrequency,tb):
    arfcn=int((downFrequency-935e6)/2e5)
    if (arfcn < 0):
        print "ARFCN must be > 0 !!!"
        sys.exit(1)
    print 'ARFCN=', arfcn
    #DB modifications
    t=(arfcn,)
    conn=sqlite3.connect("/etc/OpenBTS/OpenBTS.db")
    cursor=conn.cursor()
    cursor.execute("update config set valuestring=? where keystring='GSM.Radio.C0'",t)
    conn.commit()

    #start the OpenBTS
    f=subprocess.Popen(os.path.expanduser('~/ddpOpenBTS/runOpenBTS.sh'))
    f.wait()
    tb.msgq.delete_head()
    time.sleep(0.25)
    sub_loop(tb)
              

def quitOpenBTS(downFreq, tb):
    f=subprocess.Popen(os.path.expanduser('~/ddpOpenBTS/quitOpenBTS.sh'))
    f.wait()    
    newDownFreq = getNewChannel(downFreq, tb)
    startOpenBTS(newDownFreq, tb)

        

def getNewChannel(downFreq, tb):
    newDownFreq = downFreq + 7e6
    if newDownFreq > 960e6:
        newDownFreq = 936e6

    tb.up_freq = newDownFreq - 45e6
    print "new tb.up_freq: ", tb.up_freq
    tb.msgq.delete_head()
    time.sleep(0.25)

    print 'fft size', tb.fft_size
    N = tb.fft_size
    mid = N // 2
    cusum = 0
    counter = 0
    loopcounter = 0
    

    while loopcounter < 10:

        # Get the next message sent from the C++ code (blocking call).
        # It contains the center frequency and the mag squared of the fft
        m = parse_msg(tb.msgq.delete_head())


        center_freq = m.center_freq
        bins = 102
        power_data = 0

        
        for i in range(1, bins+1):
            power_data += m.data[mid-i] + m.data[mid+i]
        power_data += m.data[mid]
        power_data /= ((2*bins) + 1)
        
        power_db = 10*math.log10(power_data/tb.usrp_rate)
        power_threshold = -70.0
        
        
        print datetime.now(), "center_freq", center_freq, "power_db", power_db
        print "precheck"

        #cusum cusum cusum is here
        cusum = max(0, cusum + power_db - power_threshold)
        loopcounter += 1
        if (cusum > 0):
            counter += 1
            if (counter > 2):
                print "CUSUM is now positive!!!"
                newDownFreq = getNewChannel(newDownFreq, tb)
                break
        else:
            counter = 0
    return newDownFreq




if __name__ == '__main__':
    t = ThreadClass()
    t.start()

    tb = my_top_block()
    try:
        tb.start()
        main_loop(tb)

    except KeyboardInterrupt:
        pass


\end{lstlisting}




\section{primaryBTS.py}
\begin{lstlisting}[language=Python]
#!/usr/bin/env python


import sys
import sqlite3
import os
import re

def main_loop():
    usage = "usage: %prog channel_freq"
    if len(sys.argv) != 2:
        print 'usage:', sys.argv[0], 'channel_freq'
        sys.exit(1)


    center_freq = int(re.match(r'\d+', sys.argv[1]).group())*1e6
    startOpenBTS(center_freq)

def startOpenBTS(frequency):            
    
    arfcn=int((frequency-935e6)/2e5)
    print 'ARFCN=', arfcn
    
    #DB modifications
    t=(arfcn,)
    conn=sqlite3.connect("/etc/OpenBTS/OpenBTS.db")
    cursor=conn.cursor()
    cursor.execute("update config set valuestring=? where keystring='GSM.Radio.C0'",t)
    conn.commit()
    
    #start the OpenBTS
    f=os.popen('~/ddpOpenBTS/runOpenBTS.sh')
    f.close()
              

if __name__ == '__main__':

    try:
        main_loop()

    except KeyboardInterrupt:
        pass

\end{lstlisting}


\section{runOpenBTS.sh}
\begin{lstlisting}[language=bash]
#!/bin/bash

sudo echo "Hi, this script starts OpenBTS in Ubuntu 12.04!"
sudo service asterisk restart
sudo gnome-terminal -x sh -c "sudo asterisk -r" &

#cd ~/OpenBTS/
#sudo gnome-terminal --tab -e "sudo smqueue/trunk/smqueue/smqueue" \
#   --tab -e "sudo subscriberRegistry/trunk/sipauthserve" &

cd ~/OpenBTS/openbts/trunk/apps/
sudo gnome-terminal --tab -e "sudo ../../../smqueue/trunk/smqueue/smqueue" \
    --tab -e "sudo ../../../subscriberRegistry/trunk/sipauthserve" &

#sudo gnome-terminal -x sh -c "sudo ./OpenBTS" &
#sudo gnome-terminal -x sh -c "sudo ./OpenBTSCLI" &
sudo gnome-terminal --tab -e "sudo ./OpenBTS" \
    --tab -e "sudo ./OpenBTSCLI" &
cd ~

\end{lstlisting}


\section{quitOpenBTS.sh}
\begin{lstlisting}[language=bash]
#!/bin/bash

sudo echo "Hi, this script turns OpenBTS off in Ubuntu 12.04!"

sudo killall transceiver smqueue sipauthserve OpenBTSCLI asterisk
\end{lstlisting}




