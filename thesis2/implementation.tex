\chapter{Implementation of cognitive radio using OpenBTS}

In our project we are able to successfully demonstrate the
coexistence of primary users and secondary users in the
GSM band. In order to accomplish this, implementation of
a cognitive radio which detects the spectrum holes in the
radio spectrum and enables secondary users to utilize these 
for communication is needed. An experimental setup is 
developed for this demonstration using OpenBTS and GNU 
radio software and USRP N2 as hardware.

Experimental setup diagram:


\section{Description of setup}
The figure above describes the experimental setup for 
two frequency system. The primary system has only
OpenBTS software and one USRP for RF front. Where as 
secondary system has OpenBTS along with GNU radio and 
two USRP kits for each of these softwares as hardware 
RF front. This secondary system has cognitive capabilities. 
To provide cognitive capabilities it was required that 
OpenBTS an GNU radio run together in the same system and 
talk to each other which was challenging. Secondary system
continuously senses the frequency band of interest and does 
decision making depending upon the analysis of the data 
collected and changes its parameters accordingly so that 
primary and secondary users coexist. The spectrum sensing 
is accomplished by using GNU radio. Also we made GNU radio 
and OpenBTS coordinate to behave in appropriate manner and 
take dynamic decisions as an when required to make over all
system behave cognitively.

First a two frequency cognitive system is developed.
For this two GSM bands are used with centre frequency
945MHz(F1) and 950MHz(F2). Secondary users are made to occupy
one of these two bands say F1. Then we make primary users 
enter the same band. This results in an increase in energy 
levels in this band which is sensed by the secondary system 
as it is continuously scanning this band. Immediately
secondary users are shifted to other frequency band (F2)
there by vacating F1 for primary users. Hence a two frequency
cognitive system demonstrating coexistence of a pair of
primary and secondary users is accomplished. 

The whole technique is described using a flow graph below:

Flow graph here :

Now this two frequency system is expanded to a 
four frequency system where we have F1=936MHz, 
F2=943MHz, F3=950MHz, F4=957MHz. The experimental setup
is also expanded with two primary systems and one secondary
system. Primary system has OpenBTS and one USRP kit and
secondary system has OpenBTS and GNU radio software and 
two USRP kits as we had previously in two frequency system.
 
Figure for 4 frequency system here:


Here we make one pair of primary users occupy one of the four
frequencies say F2. We make secondary users use one frequency
say F1. Now other pair of primary users try an enter frequency
F1 for communication. This is sensed by the secondary system 
and it tries to migrate secondary users to F2 which is also 
occupied. Our secondary system detects that F2 is occupied an
therefore continues to find spectrum hole in a four frequency 
spectrum. It finds that frequency F3 is unoccupied and thus
allows secondary users to enter F3 and utilize it for communication.
The difference between a four frequency system than a two 
frequency system is that secondary system in a four frequency
system has to first check the presence of primary users before 
switching which was not the case in two frequency system. In 
two frequency system we assumed that the other band is always 
unoccupied at the time of switching as only one pair of primary
users are there hence one band is always unoccupied.

 The following flow graph describes the four frequency cognitive system:

Flow graph here
The spectrum sensing is done by energy detection technique and
it was required that a proper threshold is set for decision making. 
A number of readings were taken to decide the noise level, energy 
level when only primary users are occupying and also energy levels 
when both primary users and secondary users exist in the same band 
for a short duration of time. The threshold value depends on the
power transmitted by the users and their distance from the USRP 
kit which is RF front for GNU radio. This distance dependency can 
be removed by setting the threshold quite low then required so 
that even if the users move far away the decision making is not
affected. 




